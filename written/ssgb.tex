\documentclass[12pt,titlepage,a4paper]{article}

\usepackage[ngerman]{babel}
\usepackage[utf8]{inputenc}
\usepackage{color}
\usepackage[a4paper,lmargin={4cm},rmargin={2cm},
tmargin={2.5cm},bmargin = {2.5cm}]{geometry}
\usepackage{amssymb}
\usepackage{amsthm}
\usepackage{graphicx}
%\usepackage{cite}
\usepackage{natbib}



%Dokument beginnt
\begin{document}
%Titlepage
\begin{titlepage}
\title{Smart Solar Growbox}
\date{2020}
\author{Jonas Valentin Rose}
\end{titlepage}
\maketitle 
\tableofcontents
\newpage
% Ab jetzt Fliesstext
\section{Vorwort}
Dies ist der schriftliche Teil meiner Maturarbeit, bei welcher ich mich  mit der Konstruktion einer smarten \& solarbetriebenen Growbox auseinandersetze. Ich stelle hier vor allem die Probleme und Fragestellungen vor, welche mich während den einzelnen Phasen beschäftigten. Darauf folgen gelungene und gescheiterte Lösungsansätze. Ganz am Schluss werden im Glossar Begriffe erklärt.

\section{Danksagung}
Vielen Dank Papa für deine unendlich ansteckende Begeisterung für die Materie der Technik.
\section{Der Plan}
Ich mag Petersilie, Solarzellen, Arduinos und Effizienz. Was ich nicht mag, ist es, mich jede Woche aufraffen zu müssen um meine Pflanzen zu giessen. Ausserdem möchte ich in die Ferien fahren können, ohne meinem Grossvater zur Last zu werden, da er normalerweise immer unsere Pflanzen giesst, wenn wir als Familie verreisen. Im Sommer 2019 passierte genau das, meine Cherietomaten wurden durch ihn am Leben gehalten, was ich sehr schätze. Doch es wird Zeit für eine Lösung. \\Die Lösung muss so autonom und ökologisch wie möglich sein. Ausserdem soll sie Pflanzen schnell wachsen lassen können, sie muss also gute Wachstumsbedingungen liefern. \\ Vorab kann ich schon sagen, dass etwa 85\% meines neu gelernten Wissens aus Internetportalen, welche sich mit Cannabiszüchtung beschäftigen, stammt. Auch meine Bauteile kommen praktisch ausschliesslich von 420-Onlineshops. 
\subsection{Wachstumsbedingungen}
Alle haben es in dieser Schule gelernt, die einen sogar schon im U1; Pflanzen brauchen Wasser, Luft, Licht und Nährstoffe für richtiges Wachstum. Die ersten drei sind notwendig für die Photosyntese. Ausserdem ist auch ein ädequates Klima von grosser Bedeutung; Eine Temperatur von 25-35°C und eine relative Luftfeuchtigkeit, welche sich je nach Wachstumsphase und Pflanze unterscheidet.

\subsubsection{Wasser} 
Das Wasser erhalten meine Pflanzen durch eine Technologie namens Aeroponik. Sie ist eine Unterkategorie der Hydroponik. Ich verwende einen sogenannten Fogger/Nebulizer. Durch fünf kleine Porzellanplättchen, welche mit 1.9MHz vibrieren katapultiert er Wasserteilchen in die Luft. Der daraus entstehende Nebel verteilt sich im Behälter und kann von den Wurzeln aufgenommen werden. Die Wasstertröpfchen haben im Mittel eine Grösse von fünf Mikron, also ein Zweihundertstel von einem Millimeter. Je kleiner die Tröpchen sind, desto besser werden sie von den Wurzeln aufgenommen - Fünf Mikron sind sehr klein, was diese Bewässerungsmethode äusserst effizient macht. 


\subsubsection{Licht}
Licht ist die Energieform, welche sich die Pflanze bei der Photosynthese zunutze macht, um energiearme Baustoffe in energiereiche und somit nützlichere Bausteine zu verwandeln. Auf den genauen Prozess gehe ich hier nicht ein. Da die Growbox nicht draussen steht und sie auch über das recht nützliche Attribut einer Decke verfügt, bin ich gezwungen, meine Pflanzen künstliche zu beleuchten. Nach kurzer Recherche habe ich mich für LED-Lampen entschieden; Sie sind günstig, effizient und werden somit auch nicht zu heiss.\\



%Literaturverzeichniss ins Inhaltsverzeichnis einfügen
\addcontentsline{toc}{section}{Glossar}
%\bibliography{ssgb}
\bibliographystyle{agsm}
\end{document}

